% \documentclass[draft,final]{vutinfth} % Remove option 'final' to obtain debug information.
\documentclass[draft, final]{vutinfth} % Remove option 'final' to obtain debug information.

% Extended LaTeX functionality is enables by including packages with \usepackage{...}.

\usepackage{amsmath}    % Extended typesetting of mathematical expression.
\usepackage{amssymb}    % Provides a multitude of mathematical symbols.
\usepackage{mathtools}  % Further extensions of mathematical typesetting.
\usepackage{microtype}  % Small-scale typographic enhancements.
\usepackage[inline]{enumitem} % User control over the layout of lists (itemize, enumerate, description).
\usepackage{multirow}   % Allows table elements to span several rows.
\usepackage{booktabs}   % Improves the typesettings of tables.
\usepackage{subcaption} % Allows the use of subfigures and enables their referencing.
\usepackage[ruled,linesnumbered,algochapter]{algorithm2e} % Enables the writing of pseudo code.
\usepackage[dvipsnames,table]{xcolor} % Allows the definition and use of colors. This package has to be included before tikz.
\usepackage{nag}       % Issues warnings when best practices in writing LaTeX documents are violated.
\usepackage{todonotes} % Provides tooltip-like todo notes.
\usepackage{fontspec} % Determines font encoding of the output. Font packages have to be included before this line.
\usepackage{hyperref}  % Enables cross linking in the electronic document version. This package has to be included second to last.
\usepackage[acronym,toc]{glossaries} % Enables the generation of glossaries and lists fo acronyms. This package has to be included last.

\uselanguage{english}
\bibstyle{plain}
\bibdata{main}
\setmainfont{Source Serif 4}
\setsansfont{Source Sans 3}
\setmonofont{Fira Code}
% Define convenience functions to use the author name and the thesis title in the PDF document properties.
\newcommand{\authorname}{Daniel Jonas Otto} % The author name without titles.
\newcommand{\thesistitle}{Virtual Reality as a Catalyst for the Energy Transition} % The title of the thesis. The English version should be used, if it exists.

% Set PDF document properties
\hypersetup{
    pdfpagelayout   = TwoPageRight,           % How the document is shown in PDF viewers (optional).
    linkbordercolor = {Blue},                 % The color of the borders of boxes around crosslinks (optional).
    pdfauthor       = {\authorname},          % The author's name in the document properties (optional).
    pdftitle        = {\thesistitle},         % The document's title in the document properties (optional).
    pdfsubject      = {Subject},              % The document's subject in the document properties (optional).
    pdfkeywords     = {a, list, of, keywords} % The document's keywords in the document properties (optional).
}

\setpnumwidth{2.5em}        % Avoid overfull hboxes in the table of contents (see memoir manual).
\setsecnumdepth{subsection} % Enumerate subsections.
\nonzeroparskip             % Create space between paragraphs (optional).
\setlength{\parindent}{0pt} % Remove paragraph identation (optional).
\newgeometry{top=2.5cm, bottom=2.5cm}
\OnehalfSpacing
% \sloppy
\hyphenpenalty=800
\tolerance=2000
\emergencystretch=3em

\makeindex      % Use an optional index.
% \makeglossaries % Use an optional glossary.
%\glstocfalse   % Remove the glossaries from the table of contents.

% Set persons with 4 arguments:
%  {title before name}{name}{title after name}{gender}
%  where both titles are optional (i.e. can be given as empty brackets {}).
\setauthor{}{\authorname}{}{male}
\setadvisor{Dipl.-Ing.in Dr.in}{Katharina Krösl}{}{female}

% For bachelor and master theses:
% \setfirstassistant{Pretitle}{Forename Surname}{Posttitle}{male}
% \setsecondassistant{Pretitle}{Forename Surname}{Posttitle}{male}
% \setthirdassistant{Pretitle}{Forename Surname}{Posttitle}{male}

\setregnumber{11811332}
\setdate{11}{1}{2024} % Set date with 3 arguments: {day}{month}{year}.
\settitle{\thesistitle}{Virtual Reality as a Catalyst for the Energy Transition}
% \setsubtitle{Optional Subtitle of the Thesis}{Eine wissenschaftliche Untersuchung des Potenzials von VR-Technologien in der Umweltbildung und -kommunikation}
\setthesis{bachelor}
\setcurriculum{Software and Information Engineering}{Software und Information Engineering}


\begin{document}
\frontmatter
\addtitlepage{ngerman}
\addstatementpage

\begin{acknowledgements*}
  Thank you to everyone who supported me in the creation of this work. I would especially like to thank my partner Astrid Gaderbauer for her support and understanding throughout the duration of this work. I would also like to thank Helwin Prohaska for his valuable advice and support in the development of the VR application. Additionally, I would like to thank my grandparents for their financial support and encouragement. Finally, I would like to thank my supervisor Katharina Krösl for her guidance and feedback throughout the entire work.
\end{acknowledgements*}

% \begin{kurzfassung}
% \todo{Ihr Text hier.}
% \the\baselineskip
% % print fontsize
% \the\fontdimen6\font
% \end{kurzfassung}

\begin{abstract}
This thesis presents the development of a Virtual Reality (VR) application aimed at facilitating the energy transition in Austria. It addresses the widespread perception of the energy transition as overly complex and unachievable, a view that has often led to political inaction and public uncertainty. Through the use of VR technology, this research seeks to demonstrate the necessity and feasibility of transitioning towards renewable energy sources, countering the prevailing skepticism and inertia.

The core of this study lies in creating an immersive VR experience that elucidates the concepts and advantages of the energy transition in an engaging and understandable manner. The application is designed to simplify complex environmental principles, making them accessible and relatable to users. By doing so, it aims to inspire a sense of urgency and possibility regarding the adoption of sustainable energy practices.

This work is grounded in the belief that technological innovation, particularly in the realm of VR, can play a crucial role in environmental advocacy. The goal is to empower individuals with knowledge and perspective, contributing to a more proactive and optimistic approach to environmental challenges in Austria.

The findings and developments from this thesis are expected to offer new pathways in environmental education and communication, highlighting how VR can be a potent tool in promoting understanding and action in the face of environmental challenges. This research aims to stand as a testament to the power of technology in driving societal change, especially in contexts where political leadership may be lacking.
\end{abstract}

\tableofcontents % Starred version, i.e., \tableofcontents*, removes the self-entry.
\mainmatter

\chapter{Introduction}
This bachelor thesis is dedicated to the development of an innovative Virtual Reality application, which aims to advance the energy transition in Austria. In a context where public opinion and media coverage in Austria often show a certain skepticism towards the energy transition and its challenges, this work aims to bring a positive change and deeper understanding through the use of VR technology.

In Austria, the energy transition is often perceived as complex and controversial in public debate and the news. There is an urgent need to communicate more clearly the benefits and feasibility of this transition. This bachelor thesis aims to positively influence public opinion and promote a deeper understanding of its importance by creating a VR application that conveys the principles and benefits of the energy transition in an interactive and immersive manner.

The work intends to explore the role of VR as a powerful tool in education and communication strategies, to create broader awareness and stronger support for the energy transition among the Austrian population. In doing this, I utilize the latest developments in VR technology and combine them with current environmental issues.

My goal is to make a scientifically founded contribution that considers both the technological aspects and the didactic principles in conveying the challenges and solutions of the energy transition. The work is intended to serve as a basis for future research and applications dealing with the use of VR technology in environmental education and communication.

\chapter{Related Work}

\section{Virtual Reality in Education}

A central aspect in the discussion about the use of Virtual Reality (VR) in education is addressed in the article "Virtual Reality in Education: A Tool for Learning in the Experience Age"\cite{hu2017virtual} by Radić-Weissenfeld et al. This paper offers a profound insight into the diverse applications and the potential of VR as an educational tool in today's era of experience and interactivity.

The authors discuss how VR technologies can revolutionize learning by creating immersive and interactive environments that enable learners to experience complex concepts and systems in a way that is not possible with traditional teaching methods. This approach, aimed at making abstract theories tangible through direct experience, is particularly relevant for the fundamental understanding of the energy transition and renewable energy technologies. The authors emphasize that VR technology offers the possibility to enable learners to experience the impacts of their decisions and actions in a realistic environment. This approach is especially relevant for the energy transition, as it allows learners to understand and experience the effects of their decisions on the environment.

In the context of my work, which aims to improve understanding of the functioning of renewable energy systems through VR, this article provides important insights. The principles of immersion and interactivity described in the article are fundamental pillars for the development of the VR application. In particular, the emphasis on "experience orientation" in the learning process is important for the project, as it is about giving users a deep understanding of the dynamics and impacts of various photovoltaic configurations.

Radić-Weissenfeld et al. also underline the importance of user-centricity in VR learning environments. This aspect is crucial to ensure that our VR application is not only informative but also user-friendly and engaging. By applying these principles, we can create a learning experience that is not only educational but also motivating and inspiring for users.

In conclusion, the article by Radić-Weissenfeld et al. provides an important foundation for our project by demonstrating how VR technology can be effectively used in educational contexts. The insights gained from this directly flow into the design of our VR application, with the goal of creating an intuitive and effective learning experience around the topic of energy transition.
% welche erkenntnisse konkret?

\section{VR Technology in Higher Education: New Perspectives and Applications}

Virtual Reality has established itself as a powerful tool in higher education, especially in the STEM fields (Science, Technology, Engineering, and Mathematics). According to "A Review of the Application of Virtual Reality Technology in Higher Education"\cite{ding2022review}, VR is predominantly used in these subject areas to make complex and abstract concepts more comprehensible. The immersive properties of VR enable students to not only theoretically learn technical and scientific processes but also to grasp them practically and visually, leading to a deeper understanding. This application of VR highlights the technology's potential to make challenging topics more accessible and tangible.

The empirical evidence examined in the study shows that VR has a significant impact on students' learning outcomes. The use of VR technology in higher education has been observed to improve the absorption and retention of knowledge. VR offers a unique, interactive learning experience that complements and surpasses traditional teaching methods by immersing students directly in the learning environment. This direct, experience-based learning method not only fosters an understanding of complex scientific and technical concepts but also improves long-term retention.

Furthermore, the study emphasizes that VR positively influences students' learning behavior. The incorporation of VR into curricula leads to increased engagement and motivation among students. This technology allows learners to experiment and explore in an interactive and dynamic environment, making learning an active and captivating experience. This not only arouses interest in the subject matter but also enhances problem-solving and critical analysis skills. These changes in student behavior contribute to increasing learning success and fostering a deeper, lasting interest in the topics.

\section{Significant Developments in Educational Virtual Environments}

In their groundbreaking article "Educational virtual environments: A ten-year review of empirical research (1999--2009)" \cite{mikropoulos2011educational}, Tassos A. Mikropoulos and Antonis Natsis provide a comprehensive analysis of research in the field of education-oriented virtual environments (VE) over a ten-year period. This extensive review illuminates the crucial developments and changes that virtual educational environments have undergone in the last decade, thus providing an indispensable basis for understanding the current and future trends in this field.

Mikropoulos and Natsis focus on the diverse applications of VEs in education. They note that VEs are increasingly recognized as effective tools that enrich learning through their immersive and interactive properties. Their analysis covers a broad range of application areas, from scientific subjects to complex social and cultural topics, highlighting the unique benefits that VEs offer in these various contexts.

A central aspect of their findings is the importance of interactivity in VEs. They argue that interactive elements in virtual educational environments can increase learner engagement and motivation, leading to deeper and more sustainable understanding of the learning content.

Moreover, Mikropoulos and Natsis emphasize the role of VEs in promoting collaborative learning. They demonstrate how virtual environments can support cooperative and collaborative learning, where students can jointly solve tasks and share knowledge, leading to improved understanding and higher retention. This approach is particularly effective as it emphasizes the social aspects of learning, which are often neglected in traditional learning environments.

The researchers also point out the challenges associated with implementing VEs in educational institutions. These challenges include technical limitations, high costs for setting up and maintaining such systems, and the need for adequate training for teachers to effectively use these technologies. They stress the importance of careful planning and support from educational institutions to overcome these challenges.

In conclusion, Mikropoulos and Natsis underscore the potential of VEs as transformative tools in education. They argue that these technologies, when effectively utilized, can revolutionize learning and lead to deeper, more meaningful educational experiences. Their research suggests that VEs could play a key role in shaping future educational strategies, especially in areas that require a high degree of interactivity and engagement.

Given these insights, it is clear that the work of Mikropoulos and Natsis makes an important contribution to the understanding and further development of virtual educational environments. Their research findings provide valuable insights for educational professionals, developers, and researchers working at the intersection of technology and education, forming a solid foundation for future exploration and implementation of VEs in educational contexts.

The insights of Mikropoulos and Natsis are of central importance to my project. Their paper not only offers a deep insight into the historical development and various application areas of virtual educational environments but also serves as a valuable source of inspiration for innovative teaching methods in virtual reality. The importance they place on interactivity in virtual environments is particularly relevant to my project. They provide concrete starting points for designing an interactive and engaging VR learning environment, aiming not just to entertain users but actively involve them in the learning process.

Furthermore, understanding the challenges discussed by Mikropoulos and Natsis -- particularly regarding technical, financial, and pedagogical aspects -- opens the possibility to plan and implement my project realistically and purposefully. This will be crucial for ensuring smooth integration and acceptance of the VR learning application in educational contexts.

In conclusion, the paper forms a sound basis for the development of my VR application. It serves as a compass that highlights both the potentials and pitfalls of using Virtual Reality in education. By taking these insights into account, I can develop an application that is not only technologically advanced but also pedagogically valuable, thus creating an effective and appealing learning experience for users.

\section{Effectiveness and Integration of VR-based Instruction}

The integration of VR-based instruction into traditional teaching methods is a central element of my literature review. This topic is extensively analyzed in the paper "Effectiveness of virtual reality-based instruction on students' learning outcomes in K-12 and higher education"\cite{merchant2014effectiveness}. This meta-analysis explores the effectiveness of VR in various educational contexts and details the specific conditions under which VR can significantly influence learning success. Therefore, it provides valuable insights for the effective use of VR in educational environments. The results are particularly relevant for our project due to the matching age group of the participants and the thematic focus on declarative learning objectives.

Key findings of the paper are:
\begin{itemize}
  \item The results show that VR-based educational environments significantly improve students' learning outcomes. This underscores the potential of VR as an effective tool in education.

  \item The type of learning outcomes, whether knowledge or skill acquisition, influences the effectiveness of VR-based instruction. VR is particularly effective in imparting declarative knowledge. This is relevant for our project as we focus on conveying knowledge about the functioning of photovoltaic systems and the effects of various configurations.

  \item A combination of VR and other teaching methods proves to be particularly effective. This supports an integrated teaching approach where VR is used as a complement to traditional teaching forms. This approach is particularly relevant for our project as we develop the VR application as an addition to workshops in schools.

  \item The nature of feedback and the type of learning tasks are crucial for the effectiveness of VR in an educational context. Detailed explanations are more effective for declarative tasks, while concrete feedback on the correct answer is more effective for procedural tasks. This underlines the importance of tailoring feedback to the type of learning task.

  \item The methodological rigor of study designs influences the observed learning outcomes, with higher methodological rigor leading to stronger evidence of VR's effectiveness. This highlights the importance of careful design of VR-based educational environments.

  \item The results show that VR environments in education increase learner engagement and motivation. Thus, VR serves as a tool to arouse students' interest and actively involve them in the learning process.
\end{itemize}

This paper provides valuable insights for the effective design of VR-based educational environments for my work. It emphasizes the need to carefully design VR applications to ensure they support and complement learning objectives. The insight that a combination of VR and other teaching methods, as well as tailoring feedback to the type of learning task, is crucial for the success of VR in education. These insights will be considered in developing our VR application to ensure an optimal and effective learning experience.

The effectiveness of combining VR with traditional teaching methods highlighted in this paper is already being applied in our project. Our VR application is used in school workshops to make the more abstract content tangible and comprehensible, complementing frontal teaching. This methodological connection helps regain the attention of young people and deepens their understanding of the complex topics of the energy transition. Notably, the potential influence on knowledge retention is enhanced by the interactive and immersive nature of VR.

Based on the findings from the mentioned paper and previous research, I can further develop the VR application by integrating the following elements:

\begin{itemize}
  \item Implementation of interactive feedback systems that allow learners to receive immediate and specific feedback on their actions and decisions within the VR environment. This can be realized, for example, through real-time simulation data and explanatory notes.

  \item Integration of appealing and informative data visualizations that enable users to visualize the immediate effects of their decisions on the environment and energy efficiency. Such visualizations could dynamically display changes in energy amounts, CO2 emissions, or costs depending on the chosen parameters.

  \item Development of additional interactive modules that address specific aspects of the energy transition. These could include simulations that illustrate the effects of different energy systems under selected environmental conditions.

  \item By providing various interaction options within the VR environment, different learning styles can be accommodated. This can be enabled through options for exploratory and structured learning.

  \item Ensuring that the VR application is seamlessly integrated into the preceding and subsequent content of the workshops to create a coherent learning experience. This could be achieved by providing accompanying materials and by coordinating and preparing the content with the teachers.

\end{itemize}

Although the integration of VR into traditional teaching methods is promising, we should review and adjust the effectiveness and relevance of the VR application as necessary. It is important to gather feedback from teachers and students and optimize the application accordingly to ensure that it meets learning objectives and effectively contributes to knowledge transfer.

The insights from the paper and previous studies support the further development of the VR application as an effective tool in education for the energy transition. By specifically adapting and expanding the application, we can further increase learning success and foster lasting interest in the topic. Future research and applications could focus on optimizing the interaction between VR and traditional teaching methods and examining the effects on different learning groups.


\section{VR Applications for Renewable Energy}
Works focusing on the use of VR in environmental education and awareness.

\section{Design Principles for VR}
Research dealing with User Experience (UX) and User Interface (UI) in VR environments.

\backmatter

% Use an optional list of figures.
\listoffigures % Starred version, i.e., \listoffigures*, removes the toc entry.

% Use an optional list of tables.
% \cleardoublepage % Start list of tables on the next empty right hand page.
% \listoftables % Starred version, i.e., \listoftables*, removes the toc entry.

% Use an optional list of alogrithms.
% \listofalgorithms
% \addcontentsline{toc}{chapter}{List of Algorithms}

% Add an index.
\printindex

% Add a glossary.
\printglossaries

% Add a bibliography.
\bibliographystyle{alpha}
% \bibliographystyle{plain}
\bibliography{main}

\end{document}
