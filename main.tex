\documentclass[draft,final]{vutinfth} % Remove option 'final' to obtain debug information.
% \documentclass[draft]{vutinfth} % Remove option 'final' to obtain debug information.

% Extended LaTeX functionality is enables by including packages with \usepackage{...}.
\usepackage{amsmath}    % Extended typesetting of mathematical expression.
\usepackage{amssymb}    % Provides a multitude of mathematical symbols.
\usepackage{mathtools}  % Further extensions of mathematical typesetting.
\usepackage{microtype}  % Small-scale typographic enhancements.
\usepackage[inline]{enumitem} % User control over the layout of lists (itemize, enumerate, description).
\usepackage{multirow}   % Allows table elements to span several rows.
\usepackage{booktabs}   % Improves the typesettings of tables.
\usepackage{subcaption} % Allows the use of subfigures and enables their referencing.
\usepackage[ruled,linesnumbered,algochapter]{algorithm2e} % Enables the writing of pseudo code.
\usepackage[usenames,dvipsnames,table]{xcolor} % Allows the definition and use of colors. This package has to be included before tikz.
\usepackage{nag}       % Issues warnings when best practices in writing LaTeX documents are violated.
\usepackage{todonotes} % Provides tooltip-like todo notes.
\usepackage{hyperref}  % Enables cross linking in the electronic document version. This package has to be included second to last.
\usepackage[acronym,toc]{glossaries} % Enables the generation of glossaries and lists fo acronyms. This package has to be included last.
\usepackage{fontspec} % Determines font encoding of the output. Font packages have to be included before this line.
\setmainfont{Source Serif 4}
\setsansfont{Source Sans 3}
\setmonofont[Color={0019D4}]{Fira Code}
% Define convenience functions to use the author name and the thesis title in the PDF document properties.
\newcommand{\authorname}{Daniel Jonas Otto} % The author name without titles.
\newcommand{\thesistitle}{Virtuelle Realität als Katalysator für die Energiewende} % The title of the thesis. The English version should be used, if it exists.

% Set PDF document properties
\hypersetup{
    pdfpagelayout   = TwoPageRight,           % How the document is shown in PDF viewers (optional).
    linkbordercolor = {Blue},                % The color of the borders of boxes around crosslinks (optional).
    pdfauthor       = {\authorname},          % The author's name in the document properties (optional).
    pdftitle        = {\thesistitle},         % The document's title in the document properties (optional).
    pdfsubject      = {Subject},              % The document's subject in the document properties (optional).
    pdfkeywords     = {a, list, of, keywords} % The document's keywords in the document properties (optional).
}

\setpnumwidth{2.5em}        % Avoid overfull hboxes in the table of contents (see memoir manual).
\setsecnumdepth{subsection} % Enumerate subsections.

\nonzeroparskip             % Create space between paragraphs (optional).
\setlength{\parindent}{0pt} % Remove paragraph identation (optional).

\makeindex      % Use an optional index.
\makeglossaries % Use an optional glossary.
%\glstocfalse   % Remove the glossaries from the table of contents.

% Set persons with 4 arguments:
%  {title before name}{name}{title after name}{gender}
%  where both titles are optional (i.e. can be given as empty brackets {}).
\setauthor{}{\authorname}{}{male}
\setadvisor{Dipl.-Ing.in Dr.in}{Katharina Krösl}{}{female}

% For bachelor and master theses:
% \setfirstassistant{Pretitle}{Forename Surname}{Posttitle}{male}
% \setsecondassistant{Pretitle}{Forename Surname}{Posttitle}{male}
% \setthirdassistant{Pretitle}{Forename Surname}{Posttitle}{male}

\setregnumber{11811332}
\setdate{10}{11}{2023} % Set date with 3 arguments: {day}{month}{year}.
\settitle{\thesistitle}{Virtuelle Realität als Katalysator für die Energiewende}
\setsubtitle{Optional Subtitle of the Thesis}{Eine wissenschaftliche Untersuchung des Potenzials von VR-Technologien in der Umweltbildung und -kommunikation}
\setthesis{bachelor}
\setcurriculum{Software and Information Engineering}{Software und Information Engineering}

\begin{document}
\frontmatter

\linespread{1.5}\selectfont
\addtitlepage{naustrian}
\addstatementpage

\begin{danksagung*}
\todo{Ihr Text hier.}
\end{danksagung*}

\begin{acknowledgements*}
\todo{Enter your text here.}
\end{acknowledgements*}

\begin{kurzfassung}
\todo{Ihr Text hier.}
\end{kurzfassung}

\begin{abstract}
This thesis presents the development of a Virtual Reality (VR) application aimed at facilitating the energy transition in Austria. It addresses the widespread perception of the energy transition as overly complex and unachievable, a view that has often led to political inaction and public uncertainty. Through the use of VR technology, this research seeks to demonstrate the necessity and feasibility of transitioning towards renewable energy sources, countering the prevailing skepticism and inertia.

The core of this study lies in creating an immersive VR experience that elucidates the concepts and advantages of the energy transition in an engaging and understandable manner. The application is designed to simplify complex environmental principles, making them accessible and relatable to users. By doing so, it aims to inspire a sense of urgency and possibility regarding the adoption of sustainable energy practices.

This work is grounded in the belief that technological innovation, particularly in the realm of VR, can play a crucial role in environmental advocacy. The goal is to empower individuals with knowledge and perspective, contributing to a more proactive and optimistic approach to environmental challenges in Austria.

The findings and developments from this thesis are expected to offer new pathways in environmental education and communication, highlighting how VR can be a potent tool in promoting understanding and action in the face of environmental challenges. This research aims to stand as a testament to the power of technology in driving societal change, especially in contexts where political leadership may be lacking.
\end{abstract}

\selectlanguage{naustrian}
\tableofcontents % Starred version, i.e., \tableofcontents*, removes the self-entry.
\mainmatter

\chapter{Einleitung}
Diese Bachelorarbeit widmet sich der Entwicklung einer innovativen Virtual-Reality-Anwendung, die das Ziel verfolgt, die Energiewende in Österreich voranzubringen. In einem Kontext, in dem die öffentliche Meinung und die Medienberichterstattung in Österreich häufig eine gewisse Skepsis gegenüber der Energiewende und ihren Herausforderungen aufzeigen, strebt diese Arbeit an, durch den Einsatz von VR-Technologie eine positive Veränderung zu bewirken.

In Österreich, wo die Energiewende oft in der öffentlichen Debatte und in den Nachrichten als komplexes und kontroverses Thema dargestellt wird, besteht ein dringender Bedarf, die Vorteile und die Machbarkeit eines solchen Übergangs klarer zu kommunizieren. Durch die Schaffung einer VR-Anwendung, die die Prinzipien und Vorteile der Energiewende auf eine interaktive und immersive Weise vermittelt, zielt diese Arbeit darauf ab, die öffentliche Meinung positiv zu beeinflussen und ein tieferes Verständnis für die Notwendigkeit und die Vorteile der Energiewende zu fördern.

Die Arbeit beabsichtigt, die Rolle von VR als ein mächtiges Werkzeug in der Bildungs- und Kommunikationsstrategie zu erkunden, um ein breiteres Bewusstsein und eine stärkere Unterstützung für die Energiewende in der österreichischen Bevölkerung zu schaffen. Indem sie die neuesten Entwicklungen in der VR-Technologie mit den aktuellen Diskussionen über Umweltfragen in Österreich verbindet, strebt diese Bachelorarbeit an, einen wissenschaftlich fundierten Beitrag zu leisten, der sowohl die technologischen Möglichkeiten als auch die gesellschaftlichen Herausforderungen der Energiewende beleuchtet.

\chapter{Verwandte Arbeiten}

\section{Virtual Reality in der Bildung}

Ein zentraler Aspekt in der Diskussion über den Einsatz von Virtual Reality (VR) im Bildungsbereich wird in dem Artikel "Virtual Reality in Education: A Tool for Learning in the Experience Age" von Radić-Weissenfeld, Gutl und Marenzi (2019) aufgegriffen. Dieses Paper, veröffentlicht im International Journal of Advanced Corporate Learning, bietet einen tiefgreifenden Einblick in die vielfältigen Anwendungsmöglichkeiten und das Potenzial von VR als Bildungswerkzeug in der heutigen Ära der Erfahrung und Interaktivität.

Die Autoren erörtern, wie VR-Technologien das Lernen revolutionieren können, indem sie immersive und interaktive Umgebungen schaffen, die es Lernenden ermöglichen, komplexe Konzepte und Systeme auf eine Weise zu erleben, die durch traditionelle Lehrmethoden nicht möglich ist. Dieser Ansatz, der darauf abzielt, abstrakte Theorien durch direkte Erfahrung greifbar zu machen, ist besonders relevant für das grundlegende Verständnis der Energiewende und erneuerbarer Energietechnologien.

Im Kontext meiner Arbeit, die darauf abzielt, durch VR das Verständnis für die Funktionsweise erneuerbarer Energiesysteme zu verbessern, liefert der Artikel von Radić-Weissenfeld et al. wichtige Erkenntnisse. Die in dem Artikel beschriebenen Prinzipien der Immersion und Interaktivität sind grundlegende Säulen für die Entwicklung unserer VR-Anwendung. Insbesondere die Betonung auf die "Erfahrungsorientierung" im Lernprozess ist für unser Projekt von Bedeutung, da es darum geht, den Nutzenden ein tiefes Verständnis für die Dynamik und die Auswirkungen verschiedener Photovoltaik-Konfigurationen zu vermitteln.

Radić-Weissenfeld et al. unterstreichen auch die Bedeutung der Nutzerzentrierung in VR-Lernumgebungen. Dieser Aspekt ist entscheidend, um sicherzustellen, dass unsere VR-Anwendung nicht nur informativ, sondern auch benutzerfreundlich und engagierend ist. Durch die Anwendung dieser Prinzipien können wir eine Lernerfahrung schaffen, die nicht nur lehrreich, sondern auch motivierend und inspirierend für die Nutzer ist.

Abschließend stellt der Artikel von Radić-Weissenfeld et al. eine wichtige Grundlage für unser Vorhaben dar, indem er aufzeigt, wie VR-Technologie effektiv in Bildungskontexten eingesetzt werden kann. Die daraus gewonnenen Erkenntnisse fließen direkt in die Gestaltung unserer VR-Anwendung ein, mit dem Ziel, ein intuitives und effektives Lernerlebnis rund um das Thema Energiewende zu schaffen.

\section{Effektivität und Integration von VR-basiertem Unterricht}

Die Integration von VR-basiertem Unterricht in traditionelle Lehrmethoden stellt ein zentrales Element meiner Literaturübersicht dar. Diese Thematik wird insbesondere im Paper "Effectiveness of virtual reality-based instruction on students' learning outcomes in K-12 and higher education" umfassend analysiert. Diese Meta-Analyse erforscht die Wirksamkeit von VR in verschiedenen Bildungskontexten und beleuchtet detailliert, unter welchen spezifischen Bedingungen VR den Lernerfolg signifikant beeinflussen kann. Somit bietet sie wertvolle Erkenntnisse für die effektive Nutzung von VR in Bildungsumgebungen. Die Ergebnisse sind insbesondere aufgrund der übereinstimmenden Altersgruppe der Teilnehmenden und des thematischen Fokus auf deklarative Lernziele für unser Vorhaben von hoher Relevanz.

Haupterkenntnisse des Papers sind:
\begin{itemize}
    \item Die Art der Lernergebnisse, sei es Wissens- oder Fähigkeitserwerb, beeinflusst die Effektivität von VR-basiertem Unterricht. Insbesondere zeigt sich, dass VR bei der Vermittlung von deklarativem Wissen effektiver ist.

    \item Eine Kombination von VR und anderen Lehrmethoden erweist sich als besonders effektiv. Dies bestärkt den Ansatz einer integrierten Lehrmethode, bei der VR als Ergänzung zu traditionellen Unterrichtsformen eingesetzt wird.

    \item Die Art des Feedbacks und die Natur der Lernaufgaben sind entscheidend für die Effektivität von VR im Bildungskontext. Detaillierte Erklärungen sind bei deklarativen Aufgaben wirksamer, während bei prozeduralen Aufgaben konkrete Rückmeldungen zur richtigen Antwort effektiver sind.

    \item Die methodologische Strenge der Studiendesigns beeinflusst die festgestellten Lernergebnisse, wobei höhere methodologische Strenge zu stärkeren Belegen für die Wirksamkeit von VR führt.
\end{itemize}

Für meine Arbeit bietet dieses Paper wertvolle Einblicke in die effektive Gestaltung von VR-basierten Bildungsumgebungen. Es unterstreicht die Notwendigkeit, VR-Anwendungen sorgfältig zu gestalten, um sicherzustellen, dass sie die Lernziele unterstützen und ergänzen. Besonders relevant ist die Erkenntnis, dass eine Kombination aus VR und anderen Lehrmethoden sowie die Anpassung des Feedbacks an die Art der Lernaufgabe entscheidend für den Erfolg von VR in der Bildung sind. Diese Erkenntnisse sollen in der Entwicklung unserer VR-Anwendung berücksichtigt werden, um ein optimales und effektives Lernerlebnis zu gewährleisten.

Die in diesem Paper hervorgehobene Effektivität der Kombination von VR mit traditionellen Lehrmethoden findet bereits in unserem Projekt Anwendung. Unsere VR-Anwendung wird in Schulworkshops eingesetzt, um ergänzend zum Frontalunterricht die eher abstrakten Inhalte greifbar und nachvollziehbar zu machen. Diese methodische Verbindung hilft, die Aufmerksamkeit der Jugendlichen zurückzugewinnen und vertieft das Verständnis für die komplexen Themen der Energiewende. Hervorzuheben ist dabei auch der potenzielle Einfluss auf die Wissensretention, der durch die interaktive und immersive Natur von VR verstärkt wird.

Basierend auf den Erkenntnissen aus dem genannten Paper sowie früheren Forschungen können wir unsere VR-Anwendung durch die Integration folgender Elemente weiterentwickeln:

\begin{itemize}

    \item Implementierung von interaktiven Feedback-Systemen, die es den Lernenden ermöglichen, unmittelbares und spezifisches Feedback zu ihren Aktionen und Entscheidungen innerhalb der VR-Umgebung zu erhalten. Dies lässt sich beispielsweise durch Echtzeit-Simulationsdaten und erläuternde Hinweise realisieren.

    \item Integration von ansprechenden und informativen Datenvisualisierungen, die es den Nutzenden ermöglichen, die unmittelbaren Auswirkungen ihrer Entscheidungen auf Umwelt und Energieeffizienz zu visualisieren. Solche Visualisierungen könnten dynamisch die Veränderungen in Energiemengen, CO2-Emissionen oder Kosten in Abhängigkeit von den gewählten Parametern anzeigen.

    \item Entwicklung zusätzlicher interaktiver Module, die spezifische Aspekte der Energiewende thematisieren. Diese könnten beispielsweise Simulationen umfassen, die die Auswirkungen verschiedener Energiesysteme unter ausgewählten Umweltbedingungen darstellen.

    \item Berücksichtigung unterschiedlicher Lernstile durch die Bereitstellung verschiedener Interaktionsmöglichkeiten innerhalb der VR-Umgebung. Dies könnte durch die Optionen für exploratives und strukturiertes Lernen ermöglicht werden.

    \item Sicherstellung, dass die VR-Anwendung nahtlos in die vor- und nachgelagerten Inhalte der Workshops integriert ist, um ein kohärentes Lernerlebnis zu schaffen. Dies könnte durch die Bereitstellung von Begleitmaterialien und durch die Abstimmung und Vorbereitung der Inhalte mit den Lehrkräften erreicht werden.

\end{itemize}

Obwohl die Integration von VR in traditionelle Lehrmethoden vielversprechend ist, sollten wir die Wirksamkeit und Relevanz unserer Anwendung überprüfen und wenn nötig anpassen. Es ist wichtig, Feedback von Lehrkräften und Schülern zu sammeln und die Anwendung entsprechend zu optimieren, um sicherzustellen, dass sie den Lernzielen gerecht wird und effektiv zur Wissensvermittlung beiträgt.

Die Erkenntnisse aus dem Paper und früheren Studien unterstützen die Weiterentwicklung unserer VR-Anwendung als ein wirkungsvolles Werkzeug in der Bildung zur Energiewende. Durch die gezielte Anpassung und Erweiterung der Anwendung können wir den Lernerfolg weiter steigern und ein bleibendes Interesse für die Thematik fördern. Zukünftige Forschungen und Anwendungen könnten sich darauf konzentrieren, die Interaktion zwischen VR und traditionellen Lehrmethoden weiter zu optimieren und die Effekte auf verschiedene Lerngruppen zu untersuchen.

\section{VR-Technologie in der höheren Bildung: Neue Perspektiven und Anwendungen}
Virtual Reality hat sich als einflussreiches Instrument in der Hochschulbildung, insbesondere in den STEM-Fächern (Wissenschaft, Technik, Ingenieurwesen und Mathematik), etabliert. Laut "A Review of the Application of Virtual Reality Technology in Higher Education" wird VR überwiegend in diesen Fachbereichen eingesetzt, um komplexe und abstrakte Konzepte anschaulich zu machen. Durch die immersiven Eigenschaften von VR können Studierende technische und wissenschaftliche Prozesse nicht nur theoretisch erlernen, sondern sie auch praktisch und visuell erfassen, was zu einem tieferen Verständnis führt. Diese Anwendung von VR unterstreicht das Potenzial der Technologie, anspruchsvolle Themen zugänglicher und greifbarer zu machen.

Die in der gleichen Studie untersuchten empirischen Belege zeigen, dass VR einen signifikanten Einfluss auf die Lernergebnisse der Studierenden hat. Durch den Einsatz von VR-Technologie in der Hochschulbildung wird eine Verbesserung in der Aufnahme und Retention von Wissen beobachtet. VR bietet eine einzigartige, interaktive Lernerfahrung, die traditionelle Lehrmethoden ergänzt und übertrifft, indem sie die Studierenden direkt in die Lernumgebung eintauchen lässt. Diese direkte, erfahrungsbasierte Lernmethode fördert nicht nur das Verständnis komplexer wissenschaftlicher und technischer Konzepte, sondern verbessert auch die langfristige Behaltensleistung.

Überdies wird in der Studie hervorgehoben, dass VR das Lernverhalten von Studierenden positiv beeinflusst. Der Einsatz von VR in Lehrplänen führt zu erhöhtem Engagement und Motivation unter den Studierenden. Diese Technologie ermöglicht es den Lernenden, in einer interaktiven und dynamischen Umgebung zu experimentieren und zu erkunden, was das Lernen zu einer aktiven und fesselnden Erfahrung macht. Dadurch wird nicht nur das Interesse am Stoff geweckt, sondern auch die Fähigkeit zur Problemlösung und kritischen Analyse gefördert. Diese Veränderungen im Verhalten der Studierenden tragen dazu bei, den Lernerfolg zu steigern und ein tieferes, anhaltendes Interesse an den Themen zu fördern.

\section{VR-Anwendungen für erneuerbare Energien}
Arbeiten, die sich mit dem Einsatz von VR in der Umweltbildung und -bewusstsein befassen.

\section{Designprinzipien für VR}
Forschungen, die sich mit User Experience (UX) und User Interface (UI) in VR-Umgebungen beschäftigen.

\backmatter

% Use an optional list of figures.
\listoffigures % Starred version, i.e., \listoffigures*, removes the toc entry.

% Use an optional list of tables.
\cleardoublepage % Start list of tables on the next empty right hand page.
\listoftables % Starred version, i.e., \listoftables*, removes the toc entry.

% Use an optional list of alogrithms.
\listofalgorithms
\addcontentsline{toc}{chapter}{List of Algorithms}

% Add an index.
\printindex

% Add a glossary.
\printglossaries

% Add a bibliography.
\bibliographystyle{alpha}
\bibliography{main}

\end{document}
